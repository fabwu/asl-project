\section{Introduction}\label{sec:intro}

Human perception relies heavily upon visual information, particularly images.
This medium is very powerful in gaining and conveying \textit{ideas}, memories,
and emotions. No one would doubt the saying ``A picture is worth a thousand
words''. Especially with the widespread use of the internet and the advent of
social media the ability to share and store images efficiently is crucial.

A practical way to store images without occupying too much space is to use lossy
image compression which decreases the image quality in places where a
high-quality image is not perceivable or where an image of lower quality is not
disruptive. JPEG, whose basic building block is the discrete cosine transform,
is undoubtedly the most widely used lossy compression scheme. With a good
compression quality and fast compression times, JPEG hits a sweet spot for many
practical applications.

Fractal image compression is another method for lossy image compression. The
main idea is to compress an image by exposing self-similarity which can be
observed in many places in nature (e.g. fir cones or romanesco broccoli).
Contrary to JPEG, the underlying theoretical construct of a fractal compression
scheme is that of an iterated function system (IFS).

This approach yields great
compression results in terms of compression size and quality \cite{fisher2012}.
However, it is computationally expensive to encode images because the algorithm
involves an exhaustive search over different regions of the image with many
numerical computations to find self-similarity. Optimizing these computations is
therefore crucial for an efficient implementation of the algorithm.

\mypar{Related Work} The most widely known practical fractal compression scheme
was developed and patented by Michael Barnsley and Alan Sloan in 1987. They
published a paper about their work in 1989 \cite{barnsley1989fractal}. Barnsley's
graduate student Arnaud Jacquin was the first who implemented a practical
version of it in 1992 in his PhD thesis \cite{jacquin1990fractal}. Numerous
improvements and variations have then been developed to this original approach,
e.g. archetype classification (\cite{jacobs1992image}, \cite{boss1991studies})
which decreases the exhaustive search space for self-similarity. Yuval Fisher
published a book in 1995 with a detailed description of various fractal schemes
and an elaborate list of optimizations \cite{fisher2012}.

\mypar{Contribution} Based on the explanations of Fisher in \cite{fisher2012}
and an open-source implementation written in Python \cite{github-python} and C++
\cite{github-cpp}, we implemented our own fractal image compression scheme with
quadtree partitioning and exhaustive self-similarity search. We proceeded with
improving the runtime of the compression using several performance optimizations
and vector intrinsics.

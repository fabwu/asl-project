\message{ !name(report.tex)}% IEEE standard conference template; to be used with:
%   spconf.sty  - LaTeX style file, and
%   IEEEbib.bst - IEEE bibliography style file.
% --------------------------------------------------------------------------

\documentclass[letterpaper]{article}

% custom setup

% Folder for images
\graphicspath{{images/}}

% set pdf attributes
\usepackage[pdftex,
pdftitle={A Descriptive Title, no too general, not too long},
pdfauthor={Fabian Wuethrich, Jonas Hansen, Pascal Huber},
pdfkeywords={ETH, Fastcode, ASL, Fractal Image Compression}
]{hyperref}


% set to true or false to enable or disable notes
\newboolean{shownotes}
\setboolean{shownotes}{true}

% Title.
% ------
\title{Performance Optimized Fractal Image Compression using Quadtree}
%
% Single address.
% ---------------
\name{Jonas Hansen, Pascal Huber, Fabian Wüthrich\thanks{Thanks to Janis Peyer for his contributions}}
\address{Department of Computer Science\\ ETH Zurich, Switzerland}

% For example:
% ------------
%\address{School\\
%		 Department\\
%		 Address}
%
% Two addresses (uncomment and modify for two-address case).
% ----------------------------------------------------------
%\twoauthors
%  {A. Author-one, B. Author-two\sthanks{Thanks to XYZ agency for funding.}}
%		 {School A-B\\
%		 Department A-B\\
%		 Address A-B}
%  {C. Author-three, D. Author-four\sthanks{The fourth author performed the work
%		 while at ...}}
%		 {School C-D\\
%		 Department C-D\\
%		 Address C-D}
%

\begin{document}

\message{ !name(05_conclusions.tex) !offset(-46) }
\section{Conclusions}

With scalar optimizations, our implementation gains a performance speedup
of roughly 4x, whereas the vectorized implementation has a performance speedup
of roughly 8x compared with our straightforward implementation. It is important
to mention that the runtime was decreased significantly and tangibly. Compressing
an image with $2048 \times 2048$ pixels with the baseline implementation takes
about 2 hours, whereas the vectorized implementation needs 2.5 minutes.

Our fastest implementation with a performance of 4~flops/cycle is still 4 times
below the theoretical peak performance of 16 flops/cycle. As the algorithm in
this form is not memory bound, further performance improvements can be expected.

To furthermore boost performance one would also need to apply algorithmic changes.
Using exhaustive block mapping with a rather large
domain block pool (e.g. with four rotations) is a significant performance
bottleneck which does not necessarily lead to better compression results.

We consider our optimizations to be applicable in all of these more advanced
and mature fractal image compression schemes.

\message{ !name(report.tex) !offset(26) }

\end{document}
